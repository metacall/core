
\documentclass{article}

\usepackage[pdfborder={0 0 0}]{hyperref}
\usepackage{listings}




\begin{document}

% \input{metacall_style_title.tex}

\title{MetaCall: C/C++ Code Requeriments \& Style Guidelines}

\author{Vicente Eduardo Ferrer Garcia \\ \href{mailto:vic798@gmail.com}{vic798@gmail.com}}

\maketitle

%

\begin{abstract}
\noindent
Description of code requeriments and style guidelines for MetaCall.
\end{abstract}

\setcounter{tocdepth}{2}
\tableofcontents
\listoffigures
\listoftables

\newpage

% \include{metacall_style_introduction.tex}

\section{Introduction}




%

\newpage

% \include{metacall_style_file_naming.tex}

\section{File \& Directory Tree}




%

\newpage

% \include{metacall_style_file_structure.tex}

\section{Header \& Source Files Structure}




%

\newpage

% \include{metacall_style_variable_naming.tex}

\section{Naming Convention}

\subsection{camelCase vs underscore\_case}

There is no effective reason to choose between \textbf{camelCase} or \textbf{underscore\_case} in
naming convention. But in order to be aligned with STL, Boost and Bjarne Stroustrup in
"The C++ Programming Langauge", underscore case must be always used. Upper case letters must be
avoided always except from comments or template typenames.

% \lstinputlisting[language=C]{underscore_case.cpp}

\begin{lstlisting}[language=C++]

/* Use always lower case for variables, classes ... */
int some_variable = 5;

some_class object;

/* Except from typenames in templates */

template <typename T>
constexpr T sum(const T & left, const T & right)
{
	return (left + right);
}
\end{lstlisting}

\subsection{Hungarian Notation}

Hungarian notation is an identifier naming convention in which the name of a variable or function
indicates its type or intended use.

In any programming language where you can define your own types like C/C++ this becomes useless. If you
need to determine a constrait against a variable you must define your own type. The compiler and linter
should warn against a missuse of a variable type instead of the developer by means of the variable name.

By this reason, hungarian notation must be \textbf{avoided} always.

\subsection{Class Members vs Constructors}

In order to avoid shadowing class members with constructor parameters when both are equal, an underscore
suffix must be placed after each shadowed parameter.

% \lstinputlisting[language=C]{shadow_class.hpp}

\begin{lstlisting}[language=C++]

#ifndef SHADOW_CLASS_HPP
#define SHADOW_CLASS_HPP 1

/* -- Headers -- */

#include <shadow/shadow_api.hpp>

/* -- Namespace -- */

namespace metacall_style {

/* -- Class Definitions -- */

/**
*  @brief
*    Example of shadow parameters in class constructor
*/
SHADOW_API class shadow_class
{
	public:
		/* -- Public Methods -- */

		/**
		*  @brief
		*    Shadow class constructor
		*
		*  @param[in] a
		*    Brief description of @a parameter
		*
		*  @param[in] b
		*    Brief description of @b parameter
		*
		*  @param[in] c
		*    Brief description of @c parameter
		*/
		shadow_class(int a, int b, int c);

		/**
		*  @brief
		*    Shadow class destructor
		*/
		~shadow_class(void);

	private:
		/* -- Private Member Data -- */

		int a;	/**< Brief description of @a member */
		int b;	/**< Brief description of @b member */
		int c;	/**< Brief description of @c member */
};

} /* namespace metacall_style */

#endif /* SHADOW_CLASS_HPP */

\end{lstlisting}

In order to differenciate constructor arguments from class members an underscore is placed after
parameter constructor names. By this way shadow problems with namings are avoided and the use of
class members remains easy and less cryptic.


% \lstinputlisting[language=C]{shadow_class.cpp}

\begin{lstlisting}[language=C++]

/* -- Headers -- */

#include <shadow/shadow_class.hpp>

/* -- Namespace Declarations -- */

using namespace metacall_style;

/* -- Methods -- */

shadow_class::shadow_class(int a_, int b_, int c_) : a(a_), b(b_), c(c_)
{

}

shadow_class::~shadow_class()
{

}

\end{lstlisting}

%


\newpage

% \include{metacall_style_indentation.tex}

\section{Indentation \& Brackets}




%


\newpage

% \include{metacall_style_comments.tex}

\section{Comments}




%

\end{document}
